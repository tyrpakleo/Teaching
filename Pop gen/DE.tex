\documentclass[../main.tex]{subfiles}

\usepackage{graphicx} % Required for inserting images
\usepackage{appendix}
\usepackage{amsmath}
\usepackage{amssymb}
\usepackage{subfiles}
\usepackage{amsthm}

\begin{document}
\subsection{Introduction}
\subsection{Proving existence of solutions}
\subsubsection{Space of distributions}
Say we are given a differential equation and we want to prove that a solution exists in a certain space.
What's easiest is to consider the biggest space possible, where if a solution existed in any space then it would have to exist in that space.
What is this biggest space possible?
It is of course the space $\cD'(\R^d)$, the space of distributions, i.e. the dual space of $C_c^\infty(\R^d)$.

To prove a solution exists in this space we just need to define a weak form of the differential equation and then find a potential solution(which could be very irregular) and then test against functions in $C_c^\infty(\R^d)$(which are extremely regular).
\end{document}