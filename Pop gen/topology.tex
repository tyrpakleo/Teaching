\documentclass[../main.tex]{subfiles}

\usepackage{graphicx} % Required for inserting images
\usepackage{appendix}
\usepackage{amsmath}
\usepackage{amssymb}
\usepackage{subfiles}
\usepackage{amsthm}

\begin{document}
\subsection{Schwartz space}
\subsubsection{Basic introduction}
In this section we will discuss Schwartz functions, as this will allow us to better understand Schwartz distributions and all our processes take values in the set of Schwartz distributions.
Then it is essential to understand the underlying topology to proceed with the problem.

We are already familiar with the space $C_c^\infty(\R^d)$ of infinitely differentiable compactly supported functions. The topology can be defined by convergence of sequences $f_n$ converges to $f$ if and only if $f,f_n$ are all supported in the same compact set $K$ and $\nabla^jf_n\rightarrow\nabla^jf$ uniformly for all $j$, where $\nabla^j$ is the j-tensor of partial derivatives of order j.

An annoyance with this space is that it doesn't interact well with Fourier transforms, in the sense that the Fourier transform of a $C_c^\infty(\R^d)$ is not compactly supported, indeed if it were compactly supported then because the Fourier transform is real analytic by the identity theorem it is 0 everywhere. This is annoying if we would like to use any Fourier analysis on our processes.

We would still like to retain the infinite differentiability and also the convergence to $0$ at infinity, however it doesn't need to be $0$ outside a compact set which we view as too restrictive.
We say $f:\R^d\rightarrow\C$ is rapidly decreasing if $\sup_{x\in\R^d}|x|^n|f(x)|<\infty$ for all $n\in\N$. The space of Schwartz functions is infinitely differentiable functions where all derivatives are rapidly decreasing.

In similar fashion to wikipedia let's define the seminorms:
\begin{equation}
    \lVert f\rVert_{n,j}=\sup_{x\in\R^d}|x|^n|\nabla^jf(x)|
\end{equation}
Then define Schwart space as $\cS(\R^d)=\{f\in C^\infty(\R^d):\lVert f\rVert_{n,j}<\infty\forall n,j\in\N\}$
Which we endow with the topology induced by the seminorms, which turns this space into a locally convex metrizable complete topological vector space, also called Frechet space. 

We are happy as the fourier transform is a linear isomorphism from the Schwartz space to itself with the topolgy just described, which means Fourier analysis is at our fingertips if we need it.

Now unfortunately the processes with which we work are very irregular so to hope any sort of convergence result we have chosen to work on the space $\cS'(\R^d)$, the dual space of Schwartz functions which we endow with the weak-* topology.

As a quick reminder, $\cS'(\R^d)=\{\lambda:\cS(\R^d)\rightarrow\C|\lambda$ is continuous and linear$\}$.
Say $\lambda_n\rightarrow\lambda$ iff $\forall f\in\cS(\R^d):\langle f,\lambda_n\rangle\rightarrow\langle f,\lambda\rangle$.

We call this the space of Schwartz distributions(or tempered distributions) and denote it $\cS'(\R^d)=\cS'$.
\subsubsection{Compact subsets}
We are probabilists and our ultimate aim to prove convergence in distribution of a certain process. 
As is customary in most situations, it is easier to prove that a sequence of probability distributions is compact and then to prove uniqueness of the limit by proving convergence in a separating class of functions or sets. 
The second part of the argument which is about proving convergence in a separating class is more probabilistic by nature as it requires us to understand the construction of the underlying $\sigma$ algebra.
The first part of the argument requires us to precisely get at least sufficient conditions under which a set $K\subset \cS'$ is compact
\end{document}