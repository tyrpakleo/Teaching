\documentclass{article}
\usepackage{graphicx} % Required for inserting images
\usepackage{appendix}
\usepackage{subfiles}
\usepackage{amsmath,amsthm,amssymb,latexsym} 
% For including math equations, theorems, symbols, etc
\usepackage{todonotes,comment,xr,hyperref,xcolor}


%%%%%%%%%%%%%%%%%%%%%%% NATURAL NUMBERS, INTEGERS, ETC. %%%%%%%%%%%%%%%%
\providecommand{\R}{}
\providecommand{\Z}{}
\providecommand{\N}{}
\providecommand{\C}{}
\providecommand{\Q}{}
\providecommand{\G}{}
\providecommand{\Lt}{}
\renewcommand{\R}{\mathbb{R}}
\renewcommand{\Z}{\mathbb{Z}}
\renewcommand{\N}{{\mathbb N}}
\renewcommand{\C}{\mathbb{C}}
\renewcommand{\Q}{\mathbb{Q}}
\renewcommand{\G}{\mathbb{G}}
\renewcommand{\Lt}{\mathbb{L}}
%%%%%%%%%%%%%%%%%%%%%%%%%%%%%%%%%%%%%%%%%%%%%%%%%%%%%

%%%%%%%%%%%%%%% BASIC PROBABILITY %%%%%%%%%%%%%%%%%%%%%%%%%%%%
\newcommand{\E}[1]{{\mathbf E}\left[#1\right]}										

\newcommand{\e}{{\mathbf E}}


\newcommand{\V}[1]{{\mathbf{Var}}\left\{#1\right\}}
\newcommand{\va}{{\mathbf{Var}}}
\newcommand{\p}[1]{{\mathbf P}\left\{#1\right\}}
\newcommand{\psub}[2]{{\mathbf P}_{#1}\left\{#2\right\}}
\newcommand{\psup}[2]{{\mathbf P}^{#1}\left\{#2\right\}}
\newcommand{\I}[1]{{\mathbf 1}_{[#1]}}
\newcommand{\set}[1]{\left\{ #1 \right\}}
% \Cprob Bases bracket size on term before conditioning; \probC on term after conditioning
\newcommand{\Cprob}[2]{\mathbf{P}\set{\left. #1 \; \right| \; #2}} 
\newcommand{\probC}[2]{\mathbf{P}\set{#1 \; \left|  \; #2 \right. }}
\newcommand{\phat}[1]{\ensuremath{\hat{\mathbf P}}\left\{#1\right\}}
\newcommand{\Ehat}[1]{\ensuremath{\hat{\mathbf E}}\left[#1\right]}
\newcommand{\ehat}{\ensuremath{\hat{\mathbf E}}}
\newcommand{\Esup}[2]{{\mathbf E^{#1}}\left\{#2\right\}}
\newcommand{\esup}[1]{{\mathbf E^{#1}}}
\newcommand{\Esub}[2]{{\mathbf E_{#1}}\left\{#2\right\}}
\newcommand{\esub}[1]{{\mathbf E_{#1}}}
%%%%%%%%%%%%%%%%%%%%%%%%%%%%%%%%%%%%%%%%%%%%%%%%%%%%%

%%%%%%%%%%%%%%%%%%%%%%%%%%%%% SETS %%%%%%%%%%%%%%%%%%%%%
\newcommand\cA{\mathcal A}
\newcommand\cB{\mathcal B}
\newcommand\cC{\mathcal C}
\newcommand\cD{\mathcal D}
\newcommand\cE{\mathcal E}
\newcommand\cF{\mathcal F}
\newcommand\cG{\mathcal G}
\newcommand\cH{\mathcal H}
\newcommand\cI{\mathcal I}
\newcommand\cJ{\mathcal J}
\newcommand\cK{\mathcal K}
\newcommand\cL{{\mathcal L}}
\newcommand\cM{\mathcal M}
\newcommand\cN{\mathcal N}
\newcommand\cO{\mathcal O}
\newcommand\cP{\mathcal P}
\newcommand\cQ{\mathcal Q}
\newcommand\cR{{\mathcal R}}
\newcommand\cS{{\mathcal S}}
\newcommand\cT{{\mathcal T}}
\newcommand\cU{{\mathcal U}}
\newcommand\cV{\mathcal V}
\newcommand\cW{\mathcal W}
\newcommand\cX{{\mathcal X}}
\newcommand\cY{{\mathcal Y}}
\newcommand\cZ{{\mathcal Z}}
%%%%%%%%%%%%%%%%%%%%%%%%%%%%%%%%%%%%%%%%%%%%%%%%%%%%%%

%%%%%%%%%%%%%%%%%%%%%%%%%%%%% BOLDFACE %%%%%%%%%%%%%%%%%%%%
\newcommand{\bA}{\mathbf{A}} 
\newcommand{\bB}{\mathbf{B}} 
\newcommand{\bC}{\mathbf{C}} 
\newcommand{\bD}{\mathbf{D}} 
\newcommand{\bE}{\mathbf{E}} 
\newcommand{\bF}{\mathbf{F}} 
\newcommand{\bG}{\mathbf{G}} 
\newcommand{\bH}{\mathbf{H}} 
\newcommand{\bI}{\mathbf{I}} 
\newcommand{\bJ}{\mathbf{J}} 
\newcommand{\bK}{\mathbf{K}} 
\newcommand{\bL}{\mathbf{L}} 
\newcommand{\bM}{\mathbf{M}} 
\newcommand{\bN}{\mathbf{N}} 
\newcommand{\bO}{\mathbf{O}} 
\newcommand{\bP}{\mathbf{P}} 
\newcommand{\bQ}{\mathbf{Q}} 
\newcommand{\bR}{\mathbf{R}} 
\newcommand{\bS}{\mathbf{S}} 
\newcommand{\bT}{\mathbf{T}} 
\newcommand{\bU}{\mathbf{U}} 
\newcommand{\bV}{\mathbf{V}} 
\newcommand{\bW}{\mathbf{W}} 
\newcommand{\bX}{\mathbf{X}} 
\newcommand{\bY}{\mathbf{Y}} 
\newcommand{\bZ}{\mathbf{Z}}
%%%%%%%%%%%%%%%%%%%%%%%%%%%%%%%%%%%%%%%%%%%%%%%%%%%%%%%%

%%%%%%%%%%%%%%% PROBABILISTIC CONVERGENCE/EQUALITY %%%%%%%%%%%%%%%%%%%%%%%
\newcommand{\eqdist}{\ensuremath{\stackrel{\mathrm{d}}{=}}}
\newcommand{\convdist}{\ensuremath{\stackrel{\mathrm{d}}{\rightarrow}}}
\newcommand{\convas}{\ensuremath{\stackrel{\mathrm{a.s.}}{\rightarrow}}}
\newcommand{\aseq}{\ensuremath{\stackrel{\mathrm{a.s.}}{=}}}


%%%%%%%%%%%%%%%%%%%%%%% Theorem types %%%%%%%%%%%%%%%%%
\newtheorem{thm}{Theorem}[section]
\newtheorem{lem}[thm]{Lemma}
\newtheorem{prop}[thm]{Proposition}
\newtheorem{cor}[thm]{Corollary}
\newtheorem{dfn}[thm]{Definition}
\newtheorem{conj}{Conjecture}
\newtheorem{ex}{Exercise}[section]
\newtheorem{claim}[thm]{Claim}
\newtheorem{cla}[thm]{Claim}
\newtheorem{remark}[thm]{Remark}
\newtheorem{hyp}[thm]{Hypothesis}
\newtheorem{notation}[thm]{Notation}
\endinput

\title{Wiener process}
\author{Leo Tyrpak}
\date{February 2025}

\begin{document}

\maketitle
\section{Construction}
What do we mean by a Wiener process?
A lot goes behind that question. 
Firstly, let's start with Brownian motion.
One of the canonical definitions of Brownian motion is as the unique process $B_t$ which satisfies the following:
\begin{itemize}
    \item $B_t-B_s$ distributed as $B_{t-s}$ independent of $\sigma(B_r:r\leq s)$
    \item $B_1\sim\cN(0,1)$
    \item $B_0=0$
    \item $B_t$ has almost surely continuous sample paths
\end{itemize}

We note that the first condition ensures that we have a Levy process and so the distribution is uniquely determined by one of the marginals. 
Since one of the marginals is normal, it is easy to see that it is normal at all rationals and so under an additional assumption of right continuity at $0$ in probability(which is implied by continuous sample paths) we get that all marginals are normal.
The condition of starting at zero is not strictly necessary, although we would usually use $x+B_t$ for a BM started at $x\neq0$.
The condition of almost surely continuous sample paths is more so a consequence of the Kolmogorov-Chentsov continuity criterion.

\section{Wiener process}
\subsection{Wiener process on Hilbert space $U$}
We can define a Wiener process to take values in a Hilbert space $U$ and then it is completely characterised by it's covariance operator $Q$ which is positive symmetric and of trace class, which means the sum of its eigenvalues is finite.
This is a much stronger condition than being bounded which just requires the eigenvalues to be bounded.
Operator norm= norm of highest eigenvalue in that case.
\subsection{Generalised Wiener process in Hilbert space}
In this case the Wiener process is a random linear map $U\ni a\rightarrow W_a(t)$ where $W_a(t)$ is a real valued stochastic process.
This can be characterised by a covariance defined as follows:
$\langle Qa,b\rangle=K(a,b)=\E{W_a(1)W_b(1)}$.
We note that $Q$ is self-adjoint and positive definite, and so spectral theorem applies so we can easily take square roots as we desire.
Let $U_0=Q^{1/2}(U)$ with the associated norm which we call the reproducing kernel.
\end{document}