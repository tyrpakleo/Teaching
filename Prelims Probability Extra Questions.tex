\documentclass{article}
\usepackage{graphicx} % Required for inserting images
\usepackage{appendix}
\usepackage{subfiles}
\usepackage{amsmath,amsthm,amssymb,latexsym} 
% For including math equations, theorems, symbols, etc
\usepackage{todonotes,comment,xr,hyperref,xcolor}


%%%%%%%%%%%%%%%%%%%%%%% NATURAL NUMBERS, INTEGERS, ETC. %%%%%%%%%%%%%%%%
\providecommand{\R}{}
\providecommand{\Z}{}
\providecommand{\N}{}
\providecommand{\C}{}
\providecommand{\Q}{}
\providecommand{\G}{}
\providecommand{\Lt}{}
\renewcommand{\R}{\mathbb{R}}
\renewcommand{\Z}{\mathbb{Z}}
\renewcommand{\N}{{\mathbb N}}
\renewcommand{\C}{\mathbb{C}}
\renewcommand{\Q}{\mathbb{Q}}
\renewcommand{\G}{\mathbb{G}}
\renewcommand{\Lt}{\mathbb{L}}
%%%%%%%%%%%%%%%%%%%%%%%%%%%%%%%%%%%%%%%%%%%%%%%%%%%%%

%%%%%%%%%%%%%%% BASIC PROBABILITY %%%%%%%%%%%%%%%%%%%%%%%%%%%%
\newcommand{\E}[1]{{\mathbf E}\left[#1\right]}										

\newcommand{\e}{{\mathbf E}}


\newcommand{\V}[1]{{\mathbf{Var}}\left\{#1\right\}}
\newcommand{\va}{{\mathbf{Var}}}
\newcommand{\p}[1]{{\mathbf P}\left\{#1\right\}}
\newcommand{\psub}[2]{{\mathbf P}_{#1}\left\{#2\right\}}
\newcommand{\psup}[2]{{\mathbf P}^{#1}\left\{#2\right\}}
\newcommand{\I}[1]{{\mathbf 1}_{[#1]}}
\newcommand{\set}[1]{\left\{ #1 \right\}}
% \Cprob Bases bracket size on term before conditioning; \probC on term after conditioning
\newcommand{\Cprob}[2]{\mathbf{P}\set{\left. #1 \; \right| \; #2}} 
\newcommand{\probC}[2]{\mathbf{P}\set{#1 \; \left|  \; #2 \right. }}
\newcommand{\phat}[1]{\ensuremath{\hat{\mathbf P}}\left\{#1\right\}}
\newcommand{\Ehat}[1]{\ensuremath{\hat{\mathbf E}}\left[#1\right]}
\newcommand{\ehat}{\ensuremath{\hat{\mathbf E}}}
\newcommand{\Esup}[2]{{\mathbf E^{#1}}\left\{#2\right\}}
\newcommand{\esup}[1]{{\mathbf E^{#1}}}
\newcommand{\Esub}[2]{{\mathbf E_{#1}}\left\{#2\right\}}
\newcommand{\esub}[1]{{\mathbf E_{#1}}}
%%%%%%%%%%%%%%%%%%%%%%%%%%%%%%%%%%%%%%%%%%%%%%%%%%%%%

%%%%%%%%%%%%%%%%%%%%%%%%%%%%% SETS %%%%%%%%%%%%%%%%%%%%%
\newcommand\cA{\mathcal A}
\newcommand\cB{\mathcal B}
\newcommand\cC{\mathcal C}
\newcommand\cD{\mathcal D}
\newcommand\cE{\mathcal E}
\newcommand\cF{\mathcal F}
\newcommand\cG{\mathcal G}
\newcommand\cH{\mathcal H}
\newcommand\cI{\mathcal I}
\newcommand\cJ{\mathcal J}
\newcommand\cK{\mathcal K}
\newcommand\cL{{\mathcal L}}
\newcommand\cM{\mathcal M}
\newcommand\cN{\mathcal N}
\newcommand\cO{\mathcal O}
\newcommand\cP{\mathcal P}
\newcommand\cQ{\mathcal Q}
\newcommand\cR{{\mathcal R}}
\newcommand\cS{{\mathcal S}}
\newcommand\cT{{\mathcal T}}
\newcommand\cU{{\mathcal U}}
\newcommand\cV{\mathcal V}
\newcommand\cW{\mathcal W}
\newcommand\cX{{\mathcal X}}
\newcommand\cY{{\mathcal Y}}
\newcommand\cZ{{\mathcal Z}}
%%%%%%%%%%%%%%%%%%%%%%%%%%%%%%%%%%%%%%%%%%%%%%%%%%%%%%

%%%%%%%%%%%%%%%%%%%%%%%%%%%%% BOLDFACE %%%%%%%%%%%%%%%%%%%%
\newcommand{\bA}{\mathbf{A}} 
\newcommand{\bB}{\mathbf{B}} 
\newcommand{\bC}{\mathbf{C}} 
\newcommand{\bD}{\mathbf{D}} 
\newcommand{\bE}{\mathbf{E}} 
\newcommand{\bF}{\mathbf{F}} 
\newcommand{\bG}{\mathbf{G}} 
\newcommand{\bH}{\mathbf{H}} 
\newcommand{\bI}{\mathbf{I}} 
\newcommand{\bJ}{\mathbf{J}} 
\newcommand{\bK}{\mathbf{K}} 
\newcommand{\bL}{\mathbf{L}} 
\newcommand{\bM}{\mathbf{M}} 
\newcommand{\bN}{\mathbf{N}} 
\newcommand{\bO}{\mathbf{O}} 
\newcommand{\bP}{\mathbf{P}} 
\newcommand{\bQ}{\mathbf{Q}} 
\newcommand{\bR}{\mathbf{R}} 
\newcommand{\bS}{\mathbf{S}} 
\newcommand{\bT}{\mathbf{T}} 
\newcommand{\bU}{\mathbf{U}} 
\newcommand{\bV}{\mathbf{V}} 
\newcommand{\bW}{\mathbf{W}} 
\newcommand{\bX}{\mathbf{X}} 
\newcommand{\bY}{\mathbf{Y}} 
\newcommand{\bZ}{\mathbf{Z}}
%%%%%%%%%%%%%%%%%%%%%%%%%%%%%%%%%%%%%%%%%%%%%%%%%%%%%%%%

%%%%%%%%%%%%%%% PROBABILISTIC CONVERGENCE/EQUALITY %%%%%%%%%%%%%%%%%%%%%%%
\newcommand{\eqdist}{\ensuremath{\stackrel{\mathrm{d}}{=}}}
\newcommand{\convdist}{\ensuremath{\stackrel{\mathrm{d}}{\rightarrow}}}
\newcommand{\convas}{\ensuremath{\stackrel{\mathrm{a.s.}}{\rightarrow}}}
\newcommand{\aseq}{\ensuremath{\stackrel{\mathrm{a.s.}}{=}}}


%%%%%%%%%%%%%%%%%%%%%%% Theorem types %%%%%%%%%%%%%%%%%
\newtheorem{thm}{Theorem}[section]
\newtheorem{lem}[thm]{Lemma}
\newtheorem{prop}[thm]{Proposition}
\newtheorem{cor}[thm]{Corollary}
\newtheorem{dfn}[thm]{Definition}
\newtheorem{conj}{Conjecture}
\newtheorem{ex}{Exercise}[section]
\newtheorem{claim}[thm]{Claim}
\newtheorem{cla}[thm]{Claim}
\newtheorem{remark}[thm]{Remark}
\newtheorem{hyp}[thm]{Hypothesis}
\newtheorem{notation}[thm]{Notation}
\endinput

\title{Prelims Probability extra questions}
\author{Leo Tyrpak}
\date{September 2024}

\begin{document}

\maketitle

\section{Functional representations}
\subsection{Moment generating functions}
Sheet 3 Question 5 asks you to prove that,
\begin{equation*}
    \E{\exp(t X)}=\exp(\lambda(e^t-1))
\end{equation*}
for $X$ a Poisson random variable of parameter $\lambda$.

The expression on the left is called a moment generating function(mgf) and is a powerful analytical tool useful in probability to get concentration inequalities and uniqueness of distributions.

We write $M_X(t)=\E{\exp(tX)}$ for the mgf at $t$.

Calculate the mgf of the following:
\begin{enumerate}
    \item Geo(p)
    \item Ber(p)
    \item Uniform on $\{1,...,n\}$
    \item Bin(n,p)
    \item Exponential rate $\lambda$
    \item Normal random variable mean $\mu$, variance $\sigma^2$
\end{enumerate}
Let $a,b$ be some constants, then what is the mgf of $Z_1=aX+b$ (i.e. $M_{Z_1}(t)$?

Now suppose $X,Y$ are independent, then what is the mgf of $Z_2=X+Y$ (i.e. $M_{Z_2}(t)$)?

You might have noticed that the mgf doesn't always exist, as sometimes the expectation of $\exp(tX)$ is infinite. 
Can you construct an example of a random variable where $\E{\exp(tX)}$ is infinite for all $t\neq0$?

By differentiating (and not worrying about the important analysis that needs to be done in the background but that you will only do later), can you relate the moments of a distribution with the mgf?

Note: a moment is an expression of the form $\E{X^k}$.
First moment is mean $\E{X}$ and second moment is $\E{X^2}$.

Use this to find expressions for the mean and variance of a random variable $X$ using the mgf.
Then check this with the distributions considered above.

Prove that for all $t$:
\begin{equation*}
    \p{X>x}\leq M_X(t)\exp(-tx)
\end{equation*}
Consider $X$ is Ber(p),Geo(p),Bin(n,p),Poi($\lambda$),Exp($\lambda$) and for each of those optimise the right hand side in $t$ to get optimal tail bounds.
\subsection{Probability generating functions}
Sheet 5 Question 4 asks you to calculate the probability generating function of a Geometric distribution.

Calculate the following pgf:
\begin{enumerate}
    \item Ber(p)
    \item Bin(n,p)
    \item Poisson($\lambda$)
    \item Uniform $\{1,...,n\}$
    \item Negative Binomial (r,p) with pdf $p_k={k+r-1 \choose k}(1-p)^kp^r$
\end{enumerate}
Let $a,b$ be some constants, then what is the pgf of $Z_1=aX+b$ (i.e. $M_{Z_1}(t)$?

Now suppose $X,Y$ are independent, then what is the pgf of $Z_2=X+Y$ (i.e. $M_{Z_2}(t)$)?

Can we recover the distribution of a random variable from the pgf?
\section{Sums of random variables}
Suppose $X_1,...,X_n$ are independent and identically distributed (i.i.d.) with $\E{X_1}=0$.
Let $S_n=X_1+...+X_n$.

Suppose $\E{|X_1|^k}<\infty$ for $k\in\N$.
What is $\E{S_n^k}$? What is $M_{S_n}(t)$ in terms of $M_{X_1}(t)$?
\section{Fun questions}
\begin{enumerate}
    \item For sheet 8 question 7, what happens if $p=a$ or $p=b$? What does the limit approach, why?
    \item Let $U_1,...,U_n$ i.i.d Uniform [0,1] random variables. What is $\p{U_1+...+U_n\leq1}$?
\end{enumerate}
\subsection{Sheet 3 Question 6}
Sheet 3 Question 6 asks us to find out how long we have to wait on average to see 2 heads in a row, then Sheet 5 Question 5 asks us to calculate the exact distribution.
\begin{enumerate}
    \item How long do you have to wait on average to get k heads in a row where $k\in\N$?
    \item What is $r_n=\p{X>n}$ where $X$ is the first time you see k heads in a row?
    \item Let Z be the first time you see $HTH$. What is $\E{Z}$ and $\p{Z>n}$?
    \item Let Z be the first time you see $HTTH$. What is $\E{Z}$?
    \item If the challenge doesn't prove enough, try to extend your methods to any sequence.
\end{enumerate}


\end{document}