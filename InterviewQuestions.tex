\documentclass{article}
\usepackage{graphicx} % Required for inserting images
\usepackage{appendix}
\usepackage{subfiles}
\usepackage{amsmath,amsthm,amssymb,latexsym} 
% For including math equations, theorems, symbols, etc
\usepackage{todonotes,comment,xr,hyperref,xcolor}


%%%%%%%%%%%%%%%%%%%%%%% NATURAL NUMBERS, INTEGERS, ETC. %%%%%%%%%%%%%%%%
\providecommand{\R}{}
\providecommand{\Z}{}
\providecommand{\N}{}
\providecommand{\C}{}
\providecommand{\Q}{}
\providecommand{\G}{}
\providecommand{\Lt}{}
\renewcommand{\R}{\mathbb{R}}
\renewcommand{\Z}{\mathbb{Z}}
\renewcommand{\N}{{\mathbb N}}
\renewcommand{\C}{\mathbb{C}}
\renewcommand{\Q}{\mathbb{Q}}
\renewcommand{\G}{\mathbb{G}}
\renewcommand{\Lt}{\mathbb{L}}
%%%%%%%%%%%%%%%%%%%%%%%%%%%%%%%%%%%%%%%%%%%%%%%%%%%%%

%%%%%%%%%%%%%%% BASIC PROBABILITY %%%%%%%%%%%%%%%%%%%%%%%%%%%%
\newcommand{\E}[1]{{\mathbf E}\left[#1\right]}										

\newcommand{\e}{{\mathbf E}}


\newcommand{\V}[1]{{\mathbf{Var}}\left\{#1\right\}}
\newcommand{\va}{{\mathbf{Var}}}
\newcommand{\p}[1]{{\mathbf P}\left\{#1\right\}}
\newcommand{\psub}[2]{{\mathbf P}_{#1}\left\{#2\right\}}
\newcommand{\psup}[2]{{\mathbf P}^{#1}\left\{#2\right\}}
\newcommand{\I}[1]{{\mathbf 1}_{[#1]}}
\newcommand{\set}[1]{\left\{ #1 \right\}}
% \Cprob Bases bracket size on term before conditioning; \probC on term after conditioning
\newcommand{\Cprob}[2]{\mathbf{P}\set{\left. #1 \; \right| \; #2}} 
\newcommand{\probC}[2]{\mathbf{P}\set{#1 \; \left|  \; #2 \right. }}
\newcommand{\phat}[1]{\ensuremath{\hat{\mathbf P}}\left\{#1\right\}}
\newcommand{\Ehat}[1]{\ensuremath{\hat{\mathbf E}}\left[#1\right]}
\newcommand{\ehat}{\ensuremath{\hat{\mathbf E}}}
\newcommand{\Esup}[2]{{\mathbf E^{#1}}\left\{#2\right\}}
\newcommand{\esup}[1]{{\mathbf E^{#1}}}
\newcommand{\Esub}[2]{{\mathbf E_{#1}}\left\{#2\right\}}
\newcommand{\esub}[1]{{\mathbf E_{#1}}}
%%%%%%%%%%%%%%%%%%%%%%%%%%%%%%%%%%%%%%%%%%%%%%%%%%%%%

%%%%%%%%%%%%%%%%%%%%%%%%%%%%% SETS %%%%%%%%%%%%%%%%%%%%%
\newcommand\cA{\mathcal A}
\newcommand\cB{\mathcal B}
\newcommand\cC{\mathcal C}
\newcommand\cD{\mathcal D}
\newcommand\cE{\mathcal E}
\newcommand\cF{\mathcal F}
\newcommand\cG{\mathcal G}
\newcommand\cH{\mathcal H}
\newcommand\cI{\mathcal I}
\newcommand\cJ{\mathcal J}
\newcommand\cK{\mathcal K}
\newcommand\cL{{\mathcal L}}
\newcommand\cM{\mathcal M}
\newcommand\cN{\mathcal N}
\newcommand\cO{\mathcal O}
\newcommand\cP{\mathcal P}
\newcommand\cQ{\mathcal Q}
\newcommand\cR{{\mathcal R}}
\newcommand\cS{{\mathcal S}}
\newcommand\cT{{\mathcal T}}
\newcommand\cU{{\mathcal U}}
\newcommand\cV{\mathcal V}
\newcommand\cW{\mathcal W}
\newcommand\cX{{\mathcal X}}
\newcommand\cY{{\mathcal Y}}
\newcommand\cZ{{\mathcal Z}}
%%%%%%%%%%%%%%%%%%%%%%%%%%%%%%%%%%%%%%%%%%%%%%%%%%%%%%

%%%%%%%%%%%%%%%%%%%%%%%%%%%%% BOLDFACE %%%%%%%%%%%%%%%%%%%%
\newcommand{\bA}{\mathbf{A}} 
\newcommand{\bB}{\mathbf{B}} 
\newcommand{\bC}{\mathbf{C}} 
\newcommand{\bD}{\mathbf{D}} 
\newcommand{\bE}{\mathbf{E}} 
\newcommand{\bF}{\mathbf{F}} 
\newcommand{\bG}{\mathbf{G}} 
\newcommand{\bH}{\mathbf{H}} 
\newcommand{\bI}{\mathbf{I}} 
\newcommand{\bJ}{\mathbf{J}} 
\newcommand{\bK}{\mathbf{K}} 
\newcommand{\bL}{\mathbf{L}} 
\newcommand{\bM}{\mathbf{M}} 
\newcommand{\bN}{\mathbf{N}} 
\newcommand{\bO}{\mathbf{O}} 
\newcommand{\bP}{\mathbf{P}} 
\newcommand{\bQ}{\mathbf{Q}} 
\newcommand{\bR}{\mathbf{R}} 
\newcommand{\bS}{\mathbf{S}} 
\newcommand{\bT}{\mathbf{T}} 
\newcommand{\bU}{\mathbf{U}} 
\newcommand{\bV}{\mathbf{V}} 
\newcommand{\bW}{\mathbf{W}} 
\newcommand{\bX}{\mathbf{X}} 
\newcommand{\bY}{\mathbf{Y}} 
\newcommand{\bZ}{\mathbf{Z}}
%%%%%%%%%%%%%%%%%%%%%%%%%%%%%%%%%%%%%%%%%%%%%%%%%%%%%%%%

%%%%%%%%%%%%%%% PROBABILISTIC CONVERGENCE/EQUALITY %%%%%%%%%%%%%%%%%%%%%%%
\newcommand{\eqdist}{\ensuremath{\stackrel{\mathrm{d}}{=}}}
\newcommand{\convdist}{\ensuremath{\stackrel{\mathrm{d}}{\rightarrow}}}
\newcommand{\convas}{\ensuremath{\stackrel{\mathrm{a.s.}}{\rightarrow}}}
\newcommand{\aseq}{\ensuremath{\stackrel{\mathrm{a.s.}}{=}}}


%%%%%%%%%%%%%%%%%%%%%%% Theorem types %%%%%%%%%%%%%%%%%
\newtheorem{thm}{Theorem}[section]
\newtheorem{lem}[thm]{Lemma}
\newtheorem{prop}[thm]{Proposition}
\newtheorem{cor}[thm]{Corollary}
\newtheorem{dfn}[thm]{Definition}
\newtheorem{conj}{Conjecture}
\newtheorem{ex}{Exercise}[section]
\newtheorem{claim}[thm]{Claim}
\newtheorem{cla}[thm]{Claim}
\newtheorem{remark}[thm]{Remark}
\newtheorem{hyp}[thm]{Hypothesis}
\newtheorem{notation}[thm]{Notation}
\endinput
\title{Interview questions}
\author{Leo Tyrpak}

\begin{document}

\maketitle
\subsection{Question 1}

A gambler has 5 pounds and he decides to play a betting game at the casino.
Every round he bets 1 pound and throws a fair 6-sided dice.
If the number on the dice is even he wins and the casino gives him back 2 pounds, he loses and gets nothing back if the number is odd.
\begin{itemize}
    \item What is the probability that the gambler wins the first game?
    \item What is the expected amount of money he has
    after 1 round?
\end{itemize}
The rounds are independent, and the gambler stops when he loses all his money or has 10 pounds and leaves happy.

\begin{itemize}
    \item What is the expected amount of money after 3 rounds?
    \item What is the probability that the gambler loses everything? \textbf{Hint: symmetry or recurrence relation}
    \item If the last question was too easy, imagine the casino creates a game where the better wins if he rolls a 1 or 2 or loses otherwise, how much should the payout be so that is the fair game when the gamble is fixed at 1 pound.
    \item In this new scenario, what is the expected number of round the better takes to get 10 pounds or lose all his money?
    \textbf{Hint: recurrence relation for expected time of winning}
\end{itemize}

\subsection{Question 2}
Imagine you have a fair dice and coin.
You throw the dice and flip the coin every round and count your total score.
First you add the score on the dice to your previous score.
If the coin shows Tails you continue for another round and if the coin shows tails you stop.
So if you throw a $5$ and flip a H then your total score is $5$.
If you throw a $3$ and flip T then you add $3$ to your total score and continue for another round
\begin{itemize}
    \item What is your expected total score in this game?
    \item What is the variance of your total score in this game? (Check they know what variance is!)
\end{itemize}

Imagine now that you play this game with just a fair dice.
You throw the dice every round and you add the result of the dice to your total score.
If the result of the dice is $1,2,3$ then you continue for another round and if the result is $4,5,6$ then you stop.
\begin{itemize}
    \item What is your expected total score in this game?
    \item What is the variance of your total score in this game?
    \item Which of the 2 games would you prefer playing?
    \item Imagine a scenario where when you roll $4,5,6$ you can either end the game and keep the number just rolled or you can flip a coin. If the coin lands H you don't keep the number you just rolled and you stop. If the coin lands T you can keep the number just rolled and roll another time. Which should you choose.
\end{itemize}

\subsection{Question 3}
There's a deck of cards and you play a game against the table.
In the first game you each take a card in turn and if you take the Queen of spades you lose, if you don't then you shuffle the cards and your friend takes a card.
\begin{itemize}
    \item If you start the game, what is the probability of winning?
\end{itemize}
Now imagine the same scenario except that when you take a card you don't put the card back and you don't shuffle the deck.
\begin{itemize}
    \item If you start the game, what is the probability of winning?
    \item In this game, does it make any difference if you can shuffle the cards after your turn of picking a card?
    \item Now imagine that you lose the game if you get a black queen and you start the game then which game should you play?
\end{itemize}
\subsection{Question 4}
\begin{itemize}
    \item You throw a fair dice four times, what is the probability that all the numbers in the throws are different?
    \item What is the probability that the numbers are in strictly increasing order?
    \item What is the probability that the numbers are in non-decreasing order?
\end{itemize}

\subsection{Question 5}
There is a group of 6 people, where some pairs of people are friends and some aren't.
Show that there are 2 people with the same number of friends in this group.
Show that there is either a group of 3 who are all friends or a group of 3 people who are all not friends.
Being friends is not transitive but is symmetric.

\subsection{Question 6}
Are you familiar with differential equations and integrals?
We have a population of foxes in England which grows over time.
We model this with a differential equation for $P(t)$ the population at time $t$,
\begin{equation}
    \frac{dP}{dt}=P, P(0)=1
\end{equation}
\begin{itemize}
    \item Solve for $P(t)$.
    \item Sketch the graph of the total population evolving over time.
    \item What does the population graph look like when $t\rightarrow\infty$?
\end{itemize}

We model this with a differential equation for $P(t)$ the population at time $t$,
\begin{equation}
    \frac{dP}{dt}=P(1-\frac{P}{K}), P(0)=1
\end{equation}
where $K>1$ represents the carrying capacity.

\begin{itemize}
    \item Solve for $P(t)$.
    \item Sketch the graph of the total population evolving over time.
    \item What does the population graph look like when $t\rightarrow\infty$? (asymptote at K)
    \item Which model seems to be most reasonable to you and why?
    \item When is population growth maximised?
\end{itemize}

\subsection{Extra small questions}
\begin{enumerate}
    \item How many zeros in $211!$
    \item What is the area of intersection of 2 circles radius $1$ centered at $(0,0)$ and $(1,0)$
    \item You are given 5 random cards, what is the probability of getting a full house (1 pair and one triple), what is the probability of getting a flush (all same suit), what is the probability of getting a straight (all in order)
\end{enumerate}

\end{document}