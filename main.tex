\documentclass{article}
\usepackage{graphicx} % Required for inserting images
\usepackage{appendix}
\usepackage{subfiles}
\usepackage{amsmath,amsthm,amssymb,latexsym} 
% For including math equations, theorems, symbols, etc
\usepackage{todonotes,comment,xr,hyperref,xcolor}


%%%%%%%%%%%%%%%%%%%%%%% NATURAL NUMBERS, INTEGERS, ETC. %%%%%%%%%%%%%%%%
\providecommand{\R}{}
\providecommand{\Z}{}
\providecommand{\N}{}
\providecommand{\C}{}
\providecommand{\Q}{}
\providecommand{\G}{}
\providecommand{\Lt}{}
\renewcommand{\R}{\mathbb{R}}
\renewcommand{\Z}{\mathbb{Z}}
\renewcommand{\N}{{\mathbb N}}
\renewcommand{\C}{\mathbb{C}}
\renewcommand{\Q}{\mathbb{Q}}
\renewcommand{\G}{\mathbb{G}}
\renewcommand{\Lt}{\mathbb{L}}
%%%%%%%%%%%%%%%%%%%%%%%%%%%%%%%%%%%%%%%%%%%%%%%%%%%%%

%%%%%%%%%%%%%%% BASIC PROBABILITY %%%%%%%%%%%%%%%%%%%%%%%%%%%%
\newcommand{\E}[1]{{\mathbf E}\left[#1\right]}										

\newcommand{\e}{{\mathbf E}}


\newcommand{\V}[1]{{\mathbf{Var}}\left\{#1\right\}}
\newcommand{\va}{{\mathbf{Var}}}
\newcommand{\p}[1]{{\mathbf P}\left\{#1\right\}}
\newcommand{\psub}[2]{{\mathbf P}_{#1}\left\{#2\right\}}
\newcommand{\psup}[2]{{\mathbf P}^{#1}\left\{#2\right\}}
\newcommand{\I}[1]{{\mathbf 1}_{[#1]}}
\newcommand{\set}[1]{\left\{ #1 \right\}}
% \Cprob Bases bracket size on term before conditioning; \probC on term after conditioning
\newcommand{\Cprob}[2]{\mathbf{P}\set{\left. #1 \; \right| \; #2}} 
\newcommand{\probC}[2]{\mathbf{P}\set{#1 \; \left|  \; #2 \right. }}
\newcommand{\phat}[1]{\ensuremath{\hat{\mathbf P}}\left\{#1\right\}}
\newcommand{\Ehat}[1]{\ensuremath{\hat{\mathbf E}}\left[#1\right]}
\newcommand{\ehat}{\ensuremath{\hat{\mathbf E}}}
\newcommand{\Esup}[2]{{\mathbf E^{#1}}\left\{#2\right\}}
\newcommand{\esup}[1]{{\mathbf E^{#1}}}
\newcommand{\Esub}[2]{{\mathbf E_{#1}}\left\{#2\right\}}
\newcommand{\esub}[1]{{\mathbf E_{#1}}}
%%%%%%%%%%%%%%%%%%%%%%%%%%%%%%%%%%%%%%%%%%%%%%%%%%%%%

%%%%%%%%%%%%%%%%%%%%%%%%%%%%% SETS %%%%%%%%%%%%%%%%%%%%%
\newcommand\cA{\mathcal A}
\newcommand\cB{\mathcal B}
\newcommand\cC{\mathcal C}
\newcommand\cD{\mathcal D}
\newcommand\cE{\mathcal E}
\newcommand\cF{\mathcal F}
\newcommand\cG{\mathcal G}
\newcommand\cH{\mathcal H}
\newcommand\cI{\mathcal I}
\newcommand\cJ{\mathcal J}
\newcommand\cK{\mathcal K}
\newcommand\cL{{\mathcal L}}
\newcommand\cM{\mathcal M}
\newcommand\cN{\mathcal N}
\newcommand\cO{\mathcal O}
\newcommand\cP{\mathcal P}
\newcommand\cQ{\mathcal Q}
\newcommand\cR{{\mathcal R}}
\newcommand\cS{{\mathcal S}}
\newcommand\cT{{\mathcal T}}
\newcommand\cU{{\mathcal U}}
\newcommand\cV{\mathcal V}
\newcommand\cW{\mathcal W}
\newcommand\cX{{\mathcal X}}
\newcommand\cY{{\mathcal Y}}
\newcommand\cZ{{\mathcal Z}}
%%%%%%%%%%%%%%%%%%%%%%%%%%%%%%%%%%%%%%%%%%%%%%%%%%%%%%

%%%%%%%%%%%%%%%%%%%%%%%%%%%%% BOLDFACE %%%%%%%%%%%%%%%%%%%%
\newcommand{\bA}{\mathbf{A}} 
\newcommand{\bB}{\mathbf{B}} 
\newcommand{\bC}{\mathbf{C}} 
\newcommand{\bD}{\mathbf{D}} 
\newcommand{\bE}{\mathbf{E}} 
\newcommand{\bF}{\mathbf{F}} 
\newcommand{\bG}{\mathbf{G}} 
\newcommand{\bH}{\mathbf{H}} 
\newcommand{\bI}{\mathbf{I}} 
\newcommand{\bJ}{\mathbf{J}} 
\newcommand{\bK}{\mathbf{K}} 
\newcommand{\bL}{\mathbf{L}} 
\newcommand{\bM}{\mathbf{M}} 
\newcommand{\bN}{\mathbf{N}} 
\newcommand{\bO}{\mathbf{O}} 
\newcommand{\bP}{\mathbf{P}} 
\newcommand{\bQ}{\mathbf{Q}} 
\newcommand{\bR}{\mathbf{R}} 
\newcommand{\bS}{\mathbf{S}} 
\newcommand{\bT}{\mathbf{T}} 
\newcommand{\bU}{\mathbf{U}} 
\newcommand{\bV}{\mathbf{V}} 
\newcommand{\bW}{\mathbf{W}} 
\newcommand{\bX}{\mathbf{X}} 
\newcommand{\bY}{\mathbf{Y}} 
\newcommand{\bZ}{\mathbf{Z}}
%%%%%%%%%%%%%%%%%%%%%%%%%%%%%%%%%%%%%%%%%%%%%%%%%%%%%%%%

%%%%%%%%%%%%%%% PROBABILISTIC CONVERGENCE/EQUALITY %%%%%%%%%%%%%%%%%%%%%%%
\newcommand{\eqdist}{\ensuremath{\stackrel{\mathrm{d}}{=}}}
\newcommand{\convdist}{\ensuremath{\stackrel{\mathrm{d}}{\rightarrow}}}
\newcommand{\convas}{\ensuremath{\stackrel{\mathrm{a.s.}}{\rightarrow}}}
\newcommand{\aseq}{\ensuremath{\stackrel{\mathrm{a.s.}}{=}}}


%%%%%%%%%%%%%%%%%%%%%%% Theorem types %%%%%%%%%%%%%%%%%
\newtheorem{thm}{Theorem}[section]
\newtheorem{lem}[thm]{Lemma}
\newtheorem{prop}[thm]{Proposition}
\newtheorem{cor}[thm]{Corollary}
\newtheorem{dfn}[thm]{Definition}
\newtheorem{conj}{Conjecture}
\newtheorem{ex}{Exercise}[section]
\newtheorem{claim}[thm]{Claim}
\newtheorem{cla}[thm]{Claim}
\newtheorem{remark}[thm]{Remark}
\newtheorem{hyp}[thm]{Hypothesis}
\newtheorem{notation}[thm]{Notation}
\endinput

\title{Teaching}
\author{Leo Tyrpak}
\date{September 2024}

\begin{document}

\maketitle

\section{Introduction}
\section{How to teach}
We note that teaching in a tutorial setting is similar to giving a lecture or a talk in many ways. 
\subsection{How to prepare for tutorials}
I should know the material extremely well and be ready to answer any reasonable questions students have about the course, problem sheets, or additional questions.
I should attempt the exercises myself before looking at the solutions so that I can see the subtleties.
I should always have an extra book with exercises so that the students don't get bored, in case they finish all the questions.
Have office hours where students can ask any questions they desire if I have an office.
\subsection{During the tutorial}
In my teaching method, the focus should be on the students.
I know the material really well or at least should make an effort to make that the case for the tutorial.
Therefore, I should let the students speak and express themselves so that they learn to explain their ideas.
Ask lots of questions to the students.
Maintain eye contact and make students interact with the class.
Make sure to be approachable, not intimidating or scary.

Write clear and structured, using steps and big ideas to clarify the solution.
Ask if they understood.
Ask if they can read my handwriting.
Address what they do incorrectly, are the details wrong or the intuitions wrong.
Explain the concepts that the solutions do not explain.
Explain common pitfalls, misconceptions, or inefficient solutions.
Is what you are doing obvious or a big leap, you can always ask.

Time management, I should make sure to cover everything in the time required.
If someone asks irrelevant questions, don't spend too long on it, maybe write it down and come back to it!

\subsection{Marking}
Ask them to hand it in person if they can, and if not, then hand it online, should be handed in 24 hours before the tutorial starts.
If for whatever reason they haven't finished the problem sheet by that time, that is fine and they should hand in what they have and then maybe attempt the rest of the sheet before the tutorial.
I should emphasise that the tutorials are there to help them, so they shouldn't stress too much before the tutorials and that we would go through the questions together.
I will be quite harsh on the first few problem sheets in presentation and mathematical rigour.
I will emphasise that mathematics should be written clearly and in such a way that it is easy for the reader to follow.
It is normal to have many drafts of solutions until one has complete and well-written solutions; this is an important task to clarify thinking.
It is important to be very mathematically rigorous at least at the beginning and then maybe less so later on.
I will mark in red.
I will hand in marked problem sheets in a timely manner, and I will focus on writing detailed comments in good handwriting.
\subsection{Expectations from students}
This is a generic statement of what I expect from students,
\begin{itemize}
    \item Hand in sheets, complete or incomplete, 24 hours before the tutorial preferably hand written but online if not.
    \item Ask plenty of questions during the tutorial, either about the questions or about the course or about applications of material, or anything in general.
    \item Be ready to present in written or orally solutions to problems.
    \item Write good mathematics, that is structured, clear and mathematically rigorous.
\end{itemize}

How to write good mathematics,
\begin{itemize}
    \item Do not hand in rough work.
    \item Explain any symbols or notation you use.
    \item Maximum 2 equals signs per line, too many and it is confusing.
    \item Use lots of space and give me space to put comments (i.e. margin or a few extra lines after a question is finished).
    \item Use good handwriting.
    \item Don't use red, that is reserved for the marker, use blue or black.
    \item Explain using words not just equations.
\end{itemize}
My notation to save time when marking sheets,
\begin{itemize}
    \item ECF- error carried forward
    \item Squiggly equals - almost right or approximately right
    \item TLDR- too long didn't read
\end{itemize}

\section{Talks}
\subsection{Preparation}

This is for preparing a presentation or a lecture,
\begin{itemize}
    \item Practice multiple times before with no audience or just 1 person in the audience.
    \item Highlight key points that the audience should know and care about.
    \item When showing proofs, sketch out the key steps of the proof, what are dirty computations, what uses the key hypotheses.
\end{itemize}

When preparing specific material from a book or lecture notes,
\begin{itemize}
    \item Write my own notes instead of reading someone else's notes.
    \item Reorder the material in a way that makes sense to me.
    \item Focus on motivation and structure.
\end{itemize}

\subsection{Presenting}
Guidance for how to present,
\begin{itemize}
    \item Use good handwriting, write big and legibly.
    \item Use different colours to make it more readable.
    \item Once I have written move out of the way so people can see what I have written.
    \item Don't immediately rub out what I wrote down, use more boards for new stuff.
    \item Be sure to ask the audience for questions and to see if they understood.
    \item Write structured and precisely with definition, lemma, theorem, etc.
    \item Be sure to introduce properly any notation or mathematical object I use before I mention it.
    \item Always say a term's name fully before using an abbreviation.
    \item Reference results to back up my claims.
\end{itemize}
\section{Courses teaching}
\subsection{Prelims probability}

\subsection{A8 Statistics}

\subsection{Numerical analysis}

\subsection{Calculus of Variations}

\subsection{Applied probability}
\end{document}
